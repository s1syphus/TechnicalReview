\documentclass{article}

\usepackage[margin=0.5in]{geometry}


\begin{document}

\title{Programming Technical Review Notes}
\author{Mike Micatka}
\date{\today}

\maketitle

\tableofcontents

\section{Data Structures}

Comparison of common data structures:

\begin{tabular}{|l|l|l|l|l|l|}

\hline

 & Linked List & Array & Dynamic Array & Balanced Tree & Hashed Array Table \\ \hline 
 Indexing & & & & & \\ \hline
 Insert/Delete at Beginning & & & & & \\ \hline
 Insert/Delete at End & & & & & \\ \hline
 Insert/Delete at Middle & & & & & \\ \hline
 Wasted Space (average) & & & & & \\ \hline
 
\end{tabular}

\subsection{Linked Lists}

\subsubsection{Description}

A linear collection of data elements (nodes) that contain contain a pointer to the next node. There are several sub-types of linked lists; singly-linked list, doubly-linked list, mulitiply-linked list, and circular-linked list among others.

\subsubsection{Pros}

\begin{itemize}
	\item Dynamic, can be grown and pruned during run-time. Easy memory allocation/deallocation
	
	\item Node insertion and deletion are easily implemented
	
	\item Linear data structures are easily created using linked-lists
	
\end{itemize}

\subsubsection{Cons}
\begin{itemize}
	\item Use more memory than arrays because of pointers needing to be stored
	
	\item Nodes must be read in order so are sequentially accessed
	
	\item Nodes are stored incontiguously so have longer access times
	
	\item Difficult to navigate backwards (doubly-linked lists are better but require an additional pointer)
	
\end{itemize}

\subsubsection{Implementation}


\subsection{Binary Trees}

\subsection{Tries}

\subsection{Stacks}

\subsection{Queues}

\subsection{Vectors and Array Lists	}

\subsection{Hash Tables}

\section{Algorithms}


\section{Concepts}


\end{document}